\documentclass{article}
\usepackage{amsmath}
\usepackage{amssymb}
\usepackage{geometry}
\geometry{a4paper, margin=2.5cm}

\title{Jugador de Hex}
\author{Lia Stephanie López Rosales}
\date{\today}

\begin{document}

\maketitle

\section{Introducción}
El jugador implementado utiliza una combinación de:
\begin{itemize}
    \item Heurísticas posicionales
    \item Búsqueda Minimax con poda alfa-beta
    \item Ajustes dinámicos tanto en las heurísticas como en la búsqueda
\end{itemize}

\section{Heurísticas}
\subsection{Peso a los Bordes (\texttt{edge\_weights})}
\textbf{Objetivo:} Favorece celdas cercanas a los bordes objetivo.\\
\textbf{Implementación:}
\begin{equation*}
    \text{edge\_weights}(i,j) = 
    \begin{cases}
        \min(y+1, size-y) & \text{jugador horizontal} \\
        \min(x+1, size-x) & \text{jugador vertical}
    \end{cases}
\end{equation*}

\textbf{Efectividad:} Controlar celdas adyacentes a los bordes propios permite construir puentes ganadores posteriormente.

\subsection{Peso al Centro (\texttt{center\_weights})}
\textbf{Objetivo:} Valora posiciones centrales estratégicas.\\
\textbf{Implementación:}
\begin{equation*}
    \text{center\_weights}(i,j) = \frac{1}{1 + \sqrt{(i-c)^2 + (j-c)^2}},\quad c = \frac{size-1}{2}
\end{equation*}

\textbf{Efectividad:} Una celda central tiene 6 vecinos vs 3 en bordes, maximizando opciones de expansión.

\subsection{Bonificación por Puentes (\texttt{bridge\_bonus})}
\textbf{Objetivo:} Incentiva conexiones diagonales que forman "puentes".\\
\textbf{Implementación:}
\begin{equation*}
    \text{bridge\_bonus}(i,j) = 
    \begin{cases}
        2 & (i+j) \equiv 0 \mod 2 \\
        1 & \text{otherwise}
    \end{cases}
\end{equation*}

\textbf{Efectividad:} Dos piezas en puente crean caminos duales $\Rightarrow$ inmunidad a bloqueos simples.

\subsection{Penalización por Cercanía al Oponente (\texttt{opponent\_penalty})}
\textbf{Objetivo:} Desincentivar la colocación de piezas en zonas controladas por el oponente.\\
\textbf{Implementación:}
\begin{equation*}
    \text{opponent\_penalty}(i,j) = 1.5 - 0.5 \cdot \frac{|i - j|}{\text{size}}
\end{equation*}

\textbf{Efectividad:} Evitar intentar conectar bpor una zona fácilmente bloqueable o donde ya no es posible conectar

\section{Función de Evaluación}
Combina componentes con pesos adaptativos:
\begin{equation*}
    \text{Valor} = 0.6P + 0.3C + 0.1A + 0.3M
\end{equation*}
donde:
\begin{align*}
    P & = \text{Control posicional (Bordes)} \\
    C & = \text{Conexiones potenciales(Bonificación por Puentes)} \\
    A & = \text{Control de área(Penalización por Cercanía al Oponente)} \\
    M & = \text{Influencia central}
\end{align*}

\section{Ordenamiento de Movimientos}
Algoritmo de priorización:
\begin{enumerate}
    \item Calcular score por celda: 
    \begin{equation*}
        \text{score} = \underbrace{w_e}_{\text{borde}} + \underbrace{w_c}_{\text{centro}} + \underbrace{0.8w_b}_{\text{puente}} - \underbrace{0.6w_o}_{\text{oponente}}
    \end{equation*}
    \item Ordenar descendente
    \item Podar el 40\% inferior en fase tardía
\end{enumerate}

\section{Profundidad Iterativa}
Para simultáneamente no incumplir con lar restricciones de tiempo y aprovecharlo todo lo posible, se implementa un sistema que evalua antes de aumentar la profundidad de búsqueda el tiempo disponible, devolviendo la mejor jugada alcanzada según el tiempo dado en vez de tener una profundidad prefijada.

\end{document}